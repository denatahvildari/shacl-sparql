% \documentclass[10pt, conference, compsocconf]{IEEEtran}
\documentclass[]{article}


\usepackage{xspace}

\usepackage{amsfonts}
\usepackage{amsmath}
\usepackage{amsthm}
\usepackage{amssymb}
\usepackage{mathtools}

\usepackage{csquotes}

\usepackage[defblank]{paralist}
\usepackage{booktabs}

\usepackage{graphicx}
\usepackage{color}

\usepackage{url}


\usepackage{xcolor}
\usepackage{enumitem}
\usepackage{fancyvrb}

\input{../../../latex/macros/ogi-macros2}
\input{../../../latex/macros/ogi-magik-macros}
\input{../../../latex/macros/shapes-macros}

 \newcommand{\exFont}[1]{\texttt{{\small #1}}}
\begin{document}

\title{JSON syntax for input \shacl shapes}
\date{}
\maketitle

\section{Example}


The following JSON object represents the shape ``\exFont{MovieShape}'':
% \framebox[1.1\width]{
\begin{Verbatim}[frame=single]
{
  "name": "MovieShape",
  "targetDef":{
    "query":"SELECT ?x WHERE {?x a dbo:Film}"
  },
  "constraintDef":{
    "conjunctions":[
      [ 	
        { "path": "dbo:title", "min": 1, "max":1},
          { "path": "dbo:imdbId", "min": 1, datatype:"xsd:int"},
          { "path": "dbo:imdbId", "max":1},
          { "path": "dbo:director", "min": 1,
            "shape":"DirectorShape"},
          { "path": "dbo:starring", "max": 0,
            "shape":"ActorShape", "negated":true},
      ]
    ]
  }
}
\end{Verbatim}
%}

Intuitively, this shape:
\begin{itemize}
\item has all instances of ``\exFont{dbo:Film}'' as targets,
\item is verified by a node if:
\begin{itemize}
\item it has exactly one title
\item it has exactly one imdb identifier, which is an integer
\item it has a director that verifies the shape ``\exFont{DirectorShape}''
\item all its actors verify the shape ``\exFont{ActorShape}''
\end{itemize}
\end{itemize}



\section{Fields}



The (mandatory) field ``\exFont{name}'' contains the name of the shape.

The (optional) field ``\exFont{targetDef.query}'' contains a monadic \sparql query,
in charge of retrieving the targets of the shape.
The target must be bound to variable \exFont{?x}.
A shape $s$ without target definition is considered to have no target (i.e. $\tDef(s) = \bot$).


The JSON object ``\exFont{constraintDef}'' defines the constraint associated to the shape (i.e. $\cDef(s)$).
In addition to the constraint presented in the submitted article, we allow specifying the datatype of a node.
To this end,
the abstract syntax for shape constraints is extended with xsd datatypes as terminal symbols.
The semantics is the one expected:
e.g. a node $v$ verifies the formula $\exFont{xsd:int}$ iff $v$ is a literal with datatype $\exFont{xsd:int}$.
Otherwise, $v$ violates $\exFont{xsd:int}$.

In the above example,
the constraint for shape \exFont{MovieShape} is:
\begin{align*}
   & (=_1 \exFont{dbo:title}.\top)\ \wedge\\
   & (\geq_1 \exFont{dbo:imdb}.\exFont{xsd:int})\ \wedge\\
   & (\leq_1 \exFont{dbo:imdb}.\top)\ \wedge\\
   & (\geq_1 \exFont{dbo:director}.\exFont{DirectorShape})\ \wedge\\
   & (\leq_0 \exFont{dbo:starring}.\neg \exFont{ActorShape})
\end{align*}

  
The constraint formula is assumed to be in disjunctive normal form,
i.e. of the form $\phi_1 \vee .. \vee \phi_n$,
where each $\phi_i$ is a conjunction of constraints.
For instance,
the formula in the above example contains only one disjunct.

Each disjunct $\phi_1$ is a conjunction of \emph{base expressions} of the form $\geq_m r.\phi'$, $\leq_n r.\phi'$ or $=_m r.\phi'$,
where $m, n \in \mathbb{N}$,
$r$ is a \sparql property path, 
and $\phi'$ is of the form $\phi''$ or $\neg \phi''$,
where $\phi''$ is either $\top$, a datatype, a constant or a shape name.

The JSON field ``\exFont{constraintDef.conjunctions}'' is an array of arrays,
one for each disjunct $\phi_i$.

The array for $\phi_i$ contains JSON objects,
one for each base expression (conjunct) of $\phi_i$.  

The JSON object for a base expression can contain the following attributes:
\begin{itemize}
\item ``\exFont{path}'': the property path $r$
\item ``\exFont{min}'': the mininmal cardinality $m$ if the expression is of the form $\geq_m r.\phi'$ or $=_m r.\phi'$
\item ``\exFont{max}'': the maximal cardinality $n$ if the expression is of the form $\leq_n r.\phi'$ or $=_n r.\phi'$
\item ``\exFont{negated}'': \exFont{true} if $\phi'$ is of the form $\neg \phi''$, and \exFont{false} (default value) otherwise
\item ``\exFont{value}'': the constant (i.e. an IRI) if $\phi''$ is a constant
\item ``\exFont{shape}'': the shape name if $\phi''$ is a shape name
\item ``\exFont{datatype}'': the xsd datatype if $\phi''$ is a datatype
\end{itemize}
Attribute \exFont{path} is mandatory.\\
At least one of ``\exFont{min}'' or ``\exFont{max}'' must be present.\\
All other attributes are optional.\\

\noindent If attribute ``\exFont{negated}'' is absent, then it has value \exFont{false} by default.\\
If none of attributes ``\exFont{value}'', ``\exFont{shape}'' or ``\exFont{datatype}'' is present,
then $\phi''$ is considered to be $\top$ by default.
\end{document}


%%% Local Variables:
%%% mode: latex
%%% TeX-master: t
%%% End:
